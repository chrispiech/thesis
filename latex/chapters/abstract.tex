When we observe thousands and sometimes millions of students learning simultaneously what patterns unfold?

Being able to autonomously understand students as they learn, both in terms of assessing knowledge and being able to provide feedback, is a grand challenge in education and one that has been studied for hundreds of years. Massive online classes, which were recently popularized, have provided a serendipitous opportunity to break ground on this important problem. In my dissertation defense I will explore data driven solutions in the domain of students learning to program -- a discipline with rich, structured assignments.

I will discuss emergent patterns in; the space of student partial solutions, in how students navigate open ended assignments and in how students work through a series of problems. Highlights include, (1) a method which yields a noteworthy improvement in the state of the art for the task of Knowledge Tracing (2) Discovery of the Poison Path pattern in how students navigate solution spaces that both: predicts how teachers would suggest a learner make forward progress and has an almost perfect logarithmic relationship with the probability of a student succeeding in the future and (3) A leading method of using deep neural networks to autonomously embed student programs into Euclidian space. This work has been featured in: Khan Academy, Coursera and Code.org.

My talk is meant to describe a nascent space of education research, and where it could go.